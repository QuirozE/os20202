% type
\documentclass{article}

% format
\usepackage[margin=1.5cm]{geometry}

% font
\renewcommand{\familydefault}{\sfdefault}
\usepackage{lmodern}

% header
\title{
    Sitemas Operativos 2020-1\\
    Tarea 1
}

% language
\usepackage[spanish]{babel}

% header
\author{
    Edgar Quiroz
}

\begin{document}
    \maketitle

    Todas las respuestas deben estar justificadas.

    \begin{enumerate}
        \item Al terminar un proceso de E/S, el dispositivo pertinente debe 
        interrumpir al CPU para indicar que los datos están listos, y el kernel
        debe recuperar estos datos. Esto es muy lento. Mencione la técnica 
        utilizada para optimizar este proceso.

        \item Sean $t_1$ y $t_2$ procesos. $t_1$ ejecuta \texttt{thread\_yield}.
        El calenderizador escoge a $t_2$ para ejecutarse. ¿Qué proceso ejecutará 
        la línea justo despuésde la llamada a \texttt{thread\_yield}?

        \item Cuando se ejecuta una rutina de interrupción, ¿siempre hay cambio
        de contexto?

        \item ¿Quién duerme a los procesos?

        \item Enumere los casos de transición de un proceso del estado 
        \texttt{running} al estado \texttt{ready}.

        \item Lo más sencillo para lidiar con una interrupción sería pasar el 
        control a un proceso especial para que determine la rutina a llamar.
        Utilice el hecho de que los tipos de interrupción están predefinidos 
        para dar una optimización a este método.

        \item ¿Cuál es la máxima cantidad de llamadas a \texttt{wait} que se 
        pueden completar sobre un semáforo de valor inicial $n$?

        \item Hay tres maneras para pasar parámetros a una llamada al sistema.
        Menciónelos. ¿Cuál es el más eficiente respecto a espacio y a tiempo?

        \item ¿En cuantas colas puede estar formado un proceso?

        \item Compara la comunicación por memoria compartida y por paso de 
        mensajes.

        \item Da dos ejemplos de instrucciones protegidas y las consecuencias en
        el caso que no lo fueran.

        \item Explica como simular la instrucción \texttt{goto} en un modelo de 
        pila.
    \end{enumerate}
\end{document}