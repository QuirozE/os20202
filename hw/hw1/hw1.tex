% type
\documentclass{article}

% format
\usepackage[margin=1.5cm]{geometry}

% font
\renewcommand{\familydefault}{\sfdefault}
\usepackage{lmodern}

% header
\title{
    Sitemas Operativos 2020-1\\
    Tarea 1
}

% language
\usepackage[spanish]{babel}

% header
\author{
    Edgar Quiroz
}

\begin{document}
    \maketitle

    Todas las respuestas deben estar justificadas.

    \begin{enumerate}
        \item Al terminar un proceso de E/S, el dispositivo pertinente debe 
        interrumpir al CPU para indicar que los datos están listos, y el kernel
        debe recuperar estos datos. Esto es muy lento. Mencione la técnica 
        utilizada para optimizar este proceso.

        Se suelen dividir a los tipos de interrupciones en dos. Los que son
        sensibles a tiempo y los que pueden esperar.

        Los primeros (como lectura de disco duro) son atentidos por un manejador
        de interrupcione de primer nivel. Este usa tablas de rutinas que se 
        ejecutan directamente. Esto las hace más rápidas, aunque más 
        complicadas y poco escalable. Sólo las rutines que realmente necesitan 
        la velocidad son atendidas por este.

        Los demás son atendidos que un manejador de segundo nivel. Este es más 
        parecido a un proceso normal. Tiene un hilo de ejecución por cada
        programa de interrupción pertinente. Esto lo hace mas robusto y 
        escalable, aunque lento.

        \item Sean $t_1$ y $t_2$ procesos. $t_1$ ejecuta \texttt{thread\_yield}.
        El calenderizador escoge a $t_2$ para ejecutarse. ¿Qué proceso ejecutará 
        la línea justo despuésde la llamada a \texttt{thread\_yield}?

        $t_2$ ejecutará la línea, puesto que las rutinas de bloqueo son 
        idénticas para todos los procesos, por lo que no son afectados por el 
        cambio de contexto. 
        
        Al terminar la rutina, $t_2$ regresaría a donde se bloqueó, puesto que 
        esta información se guarda en su contexto, que es restaurado al cambiar 
        el proceso en ejeución.

        \item Cuando se ejecuta una rutina de interrupción, ¿siempre hay cambio
        de contexto?

        Cuando se detecta una señal en el canal de interrupciones, el procesador
        guarda el estado actual y salta al manejador de interrupciones 
        en la dirección de memoria adecuada.

        El manejador determin la causa, ejecuta una acción pertienete, restaura
        el estado anterior del procesador y vuelve al punto donde ocurrió la 
        interrupción.

        Entonces siempre es necesario tener un cambio de contexto, pues es la 
        única manera de entrar y salir de la subrutina de la interrupción.

        \item ¿Quién duerme a los procesos?

        En caso de esperas para interrupciones, el proceso sabe que tiene que 
        dormir, así que se marca como no ejecutable, y pasa el control a otro 
        proceso.

        Posteriormente, algún otro proceso revisa que otro procesos están listos
        para despertar y los despierta. En cuanto haya procesador libre, 
        volverán a ser ejecutados.

        \item Enumere los casos de transición de un proceso del estado 
        \texttt{running} al estado \texttt{ready}.

        Puede ser cuando se empieza a ejecutar un proceso hijo.

        O cuando se transfiere de forma no voluntiara el control del procesador.

        O cuando se tiene que atender una interrupción (como E/S).

        \item Lo más sencillo para lidiar con una interrupción sería pasar el 
        control a un proceso especial para que determine la rutina a llamar.
        Utilice el hecho de que los tipos de interrupción están predefinidos 
        para dar una optimización a este método.

        Se utiliza una tabla de códigos de interrupción, pues son finitos, y de 
        apuntadores hacia el código que se necesita ejecutar.

        Esto elimina el proceso intermedio para determinar el tipo de 
        interrupción, haciendo todo más eficiente.

        \item ¿Cuál es la máxima cantidad de llamadas a \texttt{wait} que se 
        pueden completar sobre un semáforo de valor inicial $n$?

        Cada llamada de \texttt{wait} a lo menos disminuye en 1 el valor del 
        semáforo. Después de este punto, el lugar de disminuir el semáforo, se
        bloquean los procesos.

        Entonces, a lo más se puede llamar $n$ veces un semáforo con valor 
        inicial de $n$ si únicamente hay llamadas \texttt{wait}.

        En el caso de que haya desbloqueos \texttt{signal}, entonces habría 
        $n + numSignalCalls$ número de posibles llamadas a \texttt{wait}.

        \item Hay tres maneras para pasar parámetros a una llamada al sistema.
        Menciónelos. ¿Cuál es el más eficiente respecto a espacio y a tiempo?

        \begin{itemize}
            \item{Guardarlos en registros.}
            \item{Almacenarlos en un bloque de la memoria y pasar la dirección}
            \item{Insertarlos en el stack}
        \end{itemize}

        En espacio, son mejores los registros. 
        En tiempo, es mejor en bloque o en el stack, pues hay menos limitaciones
        sobre el acceso a ellos y a la cantidad límite que se puede guardar.

        \item ¿En cuantas colas puede estar formado un proceso?

        Un proceso entra a una cola cuando no puede continuar su ejecución por 
        un bloqueo de recursos.

        Para entrar a otra cola, debe tener otra solicitud bloqueada de 
        recursos. Y para que esto pase debe primero tener tiempo otra vez en el
        procesador.Y esto sólo pasa después de que salga de la primera cola.

        Entonces, a lo más puede estar en una cola a la vez.

        \item Compara la comunicación por memoria compartida y por paso de 
        mensajes.

        En el paso de mensajes, no se tienen problemas de bloqueo por acceso
        simultaneo. Es más sencillo conceptualmente y fácil de implementar.
        El paso de mensajes requiere llamadas a sistema. El acceso a recursos es
        más lento y sólo es factible con pocos datos. 

        En memoria compartida, el acceso es rápdipo y se pueden mantener grandes
        cantidades de datos. La consulta a memoria no requiere uso del kernel.
        Pero pueden surgir problemas al varios procesos modificar la memoria 
        simultaneamente. Así que una implementación adecuada puede ser muy 
        complicada.

        \item Da dos ejemplos de instrucciones protegidas y las consecuencias en
        el caso que no lo fueran.

        \texttt{set\_intr} permite desactivar interrupciones. En caso de no ser
        protegida, procesos de usuario podrían apropiarse del procesador sin que
        hubiera un mecanismo para removerlos.

        \texttt{switch\_thread} permite cambiar el hilo actualmente en el 
        procesador. De no ser protegida, todos los procesos podrían mandar a 
        llamara cualquier otro código, sin restricciones ni consideraciones de
        seguridad.

        \item Explica como simular la instrucción \texttt{goto} en un modelo de 
        pila.

        Se puede hacer con llamadas de cola.

        Esto es que en cada llamada a \texttt{goto}, se guarda todas las
        instrucciones faltantes de la rutina actual, y se empieza a ejecutar el 
        código de la subrutina. Esto hace que todas las llamadas a subrutinas 
        sean la última acción en cada rutina.

        No es necesario recordar el origen ni los registros.

    \end{enumerate}
\end{document}